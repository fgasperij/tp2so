\section{Sincronización}

\subsection{Liberar aula}
Subtítulo: cómo formar un grupo de 5 sin morir en el intento.

Utilizamos dos variables:
\begin{description}
  \item[grupo\_de\_salida] representa la cantidad de personas en el grupo que se está formando
  para salir. Son alumnos a los que ya se les colocó la máscara y sólo están esperando
  que se termine de formar el grupo de salida.
  \item[salieron] representa la cantidad de personas que salieron del grupo formado.
\end{description}



Me parece que permite un mayor grado de concurrencia que el número 5 afuera del aula
levante 5 tickets y que los agarren los primeros. Además, debo decrementar la cantidad de
personas dentro del aula primero, ya que una vez que se llama a la función la persona
ya alcanzó la salida del aula, la sincronización lo que permite es organizar la salida del
edificio.

Nos gustaría haber experimentado un poco más situaciones en las que están entrando personas
al aula y simultáneamente saliendo.
